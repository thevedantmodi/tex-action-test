\documentclass[12pt]{article}
\usepackage{graphicx}

\usepackage{epsfig}
\usepackage{amsmath,amsthm}
\usepackage{listings}
\usepackage{float}
\usepackage{hyperref}
\usepackage{enumitem}
\usepackage{amsfonts}
\usepackage{amssymb}

\newtheorem{lemma}{Lemma}
\newtheorem{theorem}{Theorem}

\newlength{\toppush}
\setlength{\toppush}{2\headheight}
\addtolength{\toppush}{\headsep}

\def\subjnum{COMP160}
\def\subjname{Algorithms}

\def\doheading#1#2#3{\vfill\eject\vspace*{-\toppush}%
  \vbox{\hbox to\textwidth{{\bf} \subjnum: \subjname \hfil Vedant Modi}%
    \hbox to\textwidth{{\bf} Tufts University, Spring 2024 \hfil#3\strut}%
    \hrule}}

\newcommand{\htitle}[1]{\vspace*{1.25ex plus 1ex minus 0ex}%
\begin{center}
{\large\bf #1}
\end{center}} 

\newcommand{\notimplies}{\rlap{$\quad\not$}\implies}


%%%%%%%%%%%%%%%%%%%%%%%%%%%%%%%%%%%%%%%%%%%%%%%%%%%%%%%%%%%%%%%%%%%
% BEGIN DOCUMENT
%%%%%%%%%%%%%%%%%%%%%%%%%%%%%%%%%%%%%%%%%%%%%%%%%%%%%%%%%%%%%%%%%%%
\begin{document}
\doheading{2}{title}{Important Concepts}

\begin{description}
  \item This document contains notes on important takeaways from COMP160. 
\end{description}

\section{Time Complexity}



\section{Master Method}
\subsection*{Simple}
Given a recurrence $T(n) = aT(\frac{n}{b}) + \Theta(n^d)$
\begin{itemize}
    \item $\log_b{a} > d \implies T(n) = \Theta(n^{\log_b{a}})$
    
    This is the case where the leaves dominate the asymptotic growth.
    \item $\log_b{a} = d \implies T(n) = \Theta(n^d \log n)$
    
    This is the case where each level donates equally to the asymptotic growth.
    \item $\log_b{a} < d \implies T(n) = \Theta(n^d)$
    
    This is the case where the work done at the root of the recursion tree is the greatest.
\end{itemize}

\subsection*{Complex}
Given a recurrence $T(n) = aT(\frac{n}{b}) + f(n)$ and $a,b \in \mathbb{R} \text{ s.t. } a > 0, b > 1$
\begin{itemize}
  \item $f(n) = O(n^{\log_b{a} - \epsilon})$, for some $\epsilon > 0 \implies T(n) = \Theta (n^{\log_b{a}})$
  
  \item $f(n) = \Theta(n^{\log_b{a}}) \implies T(n) = \Theta (n^{\log_b{a}}\log n)$
  
  Also, $f(n) = \Theta(n^{\log_b{a}} \log^k{n}) \implies T(n) = \Theta (n^{\log_b{a}}\log n), k \geq 0 \implies T(n) = \Theta(n^{\log_b{a}} \log^{k+1}{n})$

  \item $f(n) = \Omega(n^{\log_b{a} + \epsilon})$, for some $\epsilon > 0 \implies T(n) = \Theta (f(n))$
\end{itemize}


\end{document}
%%%%%%%%%%%%%%%%%%%%%%%%%%%%%%%%%%%%%%%%%%%%%%%%%%%%%%%%%%%%%%%%%%%%%%

